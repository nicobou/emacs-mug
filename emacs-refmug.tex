% Design for reference mug for GNU Emacs

% This design is Copyright (C) 2010, 2011 Aleksandrs Rudzitis

% Reference material is Copyright (C) 1987, 1993, 1996, 1997, 2001, 2002, 2003,
% 2004, 2005, 2006, 2007, 2008, 2009, 2010  Free Software Foundation, Inc.

% Author: Aleks Rudzitis <ajrudzit@member.fsf.org>, derived from refcard
% by Stephen Gildea <gildea@stop.mail-abuse.org>

% You can redistribute this design and/or modify
% it under the terms of the GNU General Public License as published by
% the Free Software Foundation, either version 3 of the License, or
% (at your option) any later version.

% This program is distributed in the hope that it will be useful,
% but WITHOUT ANY WARRANTY; without even the implied warranty of
% MERCHANTABILITY or FITNESS FOR A PARTICULAR PURPOSE.  See the
% GNU General Public License for more details.

% You should have received a copy of the GNU General Public License
% along with this program.  If not, see <http://www.gnu.org/licenses/>

% This file is intended to be used with the image gnu-emacs.png. This file is
% copyrighted as follows:
% Author: Luis Fernandes <elf@ee.ryerson.ca>
% Copyright (C) 2001, 2002, 2003, 2004, 2005, 2006, 2007, 2008, 2009, 2010
%  Free Software Foundation, Inc.
% License: GNU General Public License version 3 or later (see COPYING)

% To compile this design to a pdf:
% pdflatex refmug-vX.tex

\documentclass{minimal}

\usepackage[margin=0.1in,paperwidth=7.75in,paperheight=3in]{geometry}
\usepackage{multicol}
\usepackage{graphicx}

\setlength{\parindent}{0in}
\setlength{\columnsep}{20pt}
\setlength{\columnseprule}{1pt}

\begin{document}
\begin{multicols}{3}
\vspace*{\fill}
\begin{center}
\includegraphics[scale=1.]{gnu-emacs.png}
\end{center}
\vspace*{\fill}
\columnbreak
\fontsize{6pt}{7pt}\selectfont

\fontsize{8pt}{9pt}\selectfont

\def\key#1#2{#1 & \texttt{#2} \\}


\begin{tabular}{ p{1.95in} p{.5in} }
\key{help}{C-h}
\key{apropos search function}{C-h a}
\key{describe the function a key runs}{C-h k}
\key{describe function}{C-h f}
\key{describe variable}{C-h v}
\key{describe binding}{C-h b}
&\\
\key{search forward}{C-s}
\key{search backward}{C-r}
\key{switch buffer}{C-x b}
&\\
\key{run shell command}{M-!~cmd}
\key{with region contents as input}{M-|~cmd}
\key{asynchronously}{M-\&~cmd}
\key{subshell with input and output}{M-x~term}
&\\
\key{Kill rectangle}{C-x r k}
\key{Yank rectangle}{C-x r y}
\key{Insert blank space rectangle}{C-x b}
\key{Replace blank space rectangle}{C-x r c}
\end{tabular}
\begin{tabular}{ p{1.8in} p{.7in} }
\key{autocomplete word}{M-/}
&\\
\key{set mark here}{C-SPC}
\key{mark \textbf{paragraph}}{M-h}
\key{mark \textbf{function}}{C-M-h}
\key{mark entire \textbf{buffer}}{C-x h}
\key{pop to mark}{C-u C-SPC}
&\\
\key{interactive replace }{M-\%}
\key{interactive regex replace}{C-M-\%}
\key{\texttt{re-builder}}{}
&\\
\key{indent region}{C-M-\textbackslash}
\key{comment region}{M-;}
\key{\texttt{align-regexp}}{}
&\\
\key{start keyboard macro}{C-x (}
\key{end keyboard macro}{C-x )}
\key{call keyboard last macro}{C-x e}
\key{name last macro}{C-x C-k n}
\end{tabular}
\end{multicols}
\end{document}

